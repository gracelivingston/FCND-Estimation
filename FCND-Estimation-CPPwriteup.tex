\documentclass{article}
\usepackage{microtype}
\usepackage{tabu}
\usepackage{amsmath}
\usepackage{fontspec}
\usepackage{fancyhdr}
\usepackage{minted}
\usepackage{graphicx}

\pagestyle{fancy}

\title{FCND Estimation Project Writeup}
\author{Grace Livingston}

\begin{document}
\maketitle
\tableofcontents

\newpage

\section{Implementation and Tests}

\subsection{Measurement Standard Deviation}

I calculated the standard deviation of \texttt{GPSPosXY} and
\texttt{AccelXY} using Pandas on the simulator logs generated by the
\texttt{06\_SensorNoise} scenario:

\begin{minted}{python}
    import pandas as pd

    a = pd.read_csv("config/log/Graph1.txt")
    b = pd.read_csv("config/log/Graph2.txt")

    round(a.iloc(1)[1].std(), 6)
    round(b.iloc(1)[1].std(), 6)
\end{minted}

The calculated standard deviations were:

\begin{footnotesize}
\begin{center}
    \begin{tabu}to 0.6\textwidth{|[1pt]X[2$]|X[2.25$]|[1pt]}
      \tabucline[1pt]-
      {GPSPosXY} & 0.725163 \\
      \tabucline-
      {AccelXY} & 0.512859 \\
      \tabucline[1pt]-
    \end{tabu}
\end{center}
\end{footnotesize}

After this, \texttt{06\_SensorNoise} passed:

\begin{center}
    \includegraphics[width=0.75\textwidth]{06_SensorNoise.png}
\end{center}

\begin{footnotesize}
\begin{verbatim}
Simulation #12 (../config/06_SensorNoise.txt)
PASS: ABS(Quad.GPS.X-Quad.Pos.X) was less than MeasuredStdDev_GPSPosXY for 68% of the time
PASS: ABS(Quad.IMU.AX-0.000000) was less than MeasuredStdDev_AccelXY for 69% of the time
Simulation #13 (../config/06_SensorNoise.txt)
PASS: ABS(Quad.GPS.X-Quad.Pos.X) was less than MeasuredStdDev_GPSPosXY for 68% of the time
PASS: ABS(Quad.IMU.AX-0.000000) was less than MeasuredStdDev_AccelXY for 69% of the time
Simulation #14 (../config/06_SensorNoise.txt)
PASS: ABS(Quad.GPS.X-Quad.Pos.X) was less than MeasuredStdDev_GPSPosXY for 68% of the time
PASS: ABS(Quad.IMU.AX-0.000000) was less than MeasuredStdDev_AccelXY for 69% of the time
Simulation #15 (../config/06_SensorNoise.txt)
PASS: ABS(Quad.GPS.X-Quad.Pos.X) was less than MeasuredStdDev_GPSPosXY for 68% of the time
PASS: ABS(Quad.IMU.AX-0.000000) was less than MeasuredStdDev_AccelXY
for 69% of the time
\end{verbatim}
\end{footnotesize}

\newpage
\subsection{Gyro Attitude Integration}

In \texttt{UpdateFromIMU()}, I improved the gyro integration using the
rotation matrix. I also precompute the $\sin$,
$\cos$, and $\tan$ values used for the matrix. After this, the drone
could maintain attitude:


\begin{center}
    \includegraphics[width=0.75\textwidth]{07_AttitudeEstimation.png}
\end{center}

\begin{footnotesize}
\begin{verbatim}
Simulation #22 (../config/07_AttitudeEstimation.txt)
PASS: ABS(Quad.Est.E.MaxEuler) was less than 0.100000 for at least 3.000000 seconds
Simulation #23 (../config/07_AttitudeEstimation.txt)
PASS: ABS(Quad.Est.E.MaxEuler) was less than 0.100000 for at least 3.000000 seconds
Simulation #24 (../config/07_AttitudeEstimation.txt)
PASS: ABS(Quad.Est.E.MaxEuler) was less than 0.100000 for at least 3.000000 seconds
Simulation #25 (../config/07_AttitudeEstimation.txt)
PASS: ABS(Quad.Est.E.MaxEuler) was less than 0.100000 for at least 3.000000 seconds
Simulation #26 (../config/07_AttitudeEstimation.txt)
PASS: ABS(Quad.Est.E.MaxEuler) was less than 0.100000 for at least 3.000000 seconds\end{verbatim}
\end{footnotesize}

\newpage
\subsection{Prediction Step For The Estimator}

In \texttt{PredictState}, I integrate the current states for position
and velocity over the time step, transform the prediction to the
inertial frame, and then compensate for gravity:

\begin{minted}{c++}
    predictedState(0) = curState(0) + dt * curState(3);
    predictedState(1) = curState(1) + dt * curState(4);
    predictedState(2) = curState(2) + dt * curState(5);

    // Transform the acceleration to the inertial frame.
    V3F a_I = attitude.Rotate_BtoI(accel);

    // Update the predicted state with the current acceleration and
    // compensate for gravity.
    predictedState(3) = curState(3) + dt * a_I.x;
    predictedState(4) = curState(4) + dt * a_I.y;
    predictedState(5) = curState(5) + dt * a_I.z -
        dt * (float) CONST_GRAVITY;
\end{minted}

At this point the drone can fly the square in
\texttt{08\_PredictState}:

\begin{center}
    \includegraphics[width=0.75\textwidth]{08_PredictState-01.png}
\end{center}

\begin{center}
    \includegraphics[width=0.75\textwidth]{08_PredictState-02.png}
\end{center}

\begin{center}
    \includegraphics[width=0.75\textwidth]{08_PredictState-03.png}
\end{center}

\begin{center}
    \includegraphics[width=0.75\textwidth]{08_PredictState-04.png}
\end{center}

\newpage
\subsection{Magnetometer Update}

The magnetometer update is done by integrating the partial derivatives
of the magnetometer data over the prior timestep:

\begin{minted}{c++}
    gPrime(0,3) = dt;
    gPrime(1,4) = dt;
    gPrime(2,5) = dt;

    gPrime(3, 6) = (RbgPrime(0) * accel).sum() * dt;
    gPrime(4, 6) = (RbgPrime(1) * accel).sum() * dt;
    gPrime(5, 6) = (RbgPrime(2) * accel).sum() * dt;

    ekfCov = gPrime * ekfCov * gPrime.transpose() + Q;
\end{minted}

At this point the drone maintains attitude even with magnetometer
noise and passes the \texttt{10\_MagUpdate} test:

\begin{center}
    \includegraphics[width=0.75\textwidth]{10_MagUpdate-01.png}
\end{center}

\begin{center}
    \includegraphics[width=0.75\textwidth]{10_MagUpdate-02.png}
\end{center}

\newpage
\subsection{GPS Update}

Finally the GPS covariance is updated:

\begin{minted}{c++}
    zFromX(0) = ekfState(0);
    zFromX(1) = ekfState(1);
    zFromX(2) = ekfState(2);
    zFromX(3) = ekfState(3);
    zFromX(4) = ekfState(4);
    zFromX(5) = ekfState(5);

    for ( int i = 0; i < 6; i++) {
        hPrime(i,i) = 1;
    }
\end{minted}

At this point the drone manages GPS noise in \texttt{11\_GPSUpdate}:

\begin{center}
    \includegraphics[width=0.75\textwidth]{11_GPSUpdate-01.png}
\end{center}

\begin{center}
    \includegraphics[width=0.75\textwidth]{11_GPSUpdate-02.png}
\end{center}

\begin{center}
    \includegraphics[width=0.75\textwidth]{11_GPSUpdate-03.png}
\end{center}

\section{Flight Evaluation}

The \texttt{QuadPhysicalParams} had too much gain for the sensor noise
this exercise introduced. Initially I increased all of the gains to
try and increase the precision of the drone, but that ended with
several crashes, considerable instability in flight, and lots
of consternation.

After switching back to the original parameters and seeing
a noticeable improvement over the increases, I instead decreased the
gains, and as I did this the flight characteristics
improved considerably. I didn’t reduce them too much past the point
the drone could remain stable.

\end{document}
